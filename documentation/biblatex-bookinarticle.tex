\documentclass{ltxdockit}[2011/03/25]
\usepackage{btxdockit}
\usepackage{fontspec}
\usepackage[mono=false]{libertine}
\usepackage{microtype}
\usepackage[american]{babel}
\usepackage[strict]{csquotes}
\setmonofont[Scale=MatchLowercase]{DejaVu Sans Mono}
\usepackage{shortvrb}
\usepackage{pifont}
\usepackage{minted}
% Usefull commands
\newcommand{\biblatex}{biblatex\xspace}
\pretocmd{\bibfield}{\sloppy}{}{}
\pretocmd{\bibtype}{\sloppy}{}{}
\newcommand{\namebibstyle}[1]{\texttt{#1}}
% Meta-datas
\titlepage{%
	title={Book in article with biblatex},
	subtitle={New data type},
	email={maieul <at> maieul <dot> net},
	author={Maïeul Rouquette},
	revision={1.0.0},
	date={02/07/2014},
	url={https://github.com/maieul/biblatex-bookinarticle}}

% biblatex
\usepackage[citestyle=verbose]{biblatex}
\usepackage{biblatex-bookinarticle}
\addbibresource{example.bib}

\begin{document}

\printtitlepage
\tableofcontents

\section{Introduction}

In classical philology, it happens that ancient books are edited by modern scholar. So, when we refer to them, we have to not refer to the article, but, indeed, to the \emph{book which is in the article}.

This package allows to create entry which's type is `bookinarticle`, and which are printed like this:

\begin{quotation}
\cite{BHG226e}
\end{quotation}


\section{Use}

\subsection{\bibtype{bookinarticle} Entry Type}

A new entrytype is define: \bibtype{bookinarticle}. It use the standard fields of a \bibfield{article}, with those changes:

\begin{itemize}
	\item \bibfield{author} means the author of the ancient book.
	\item \textbf{\bibfield{bookauthor} means the author of the article where the book is edited, e.g. the modern editor of the book}.
	\item \bibfield{mainsubtitle} means the subtitle of the article where the book is edited.
	\item \bibfield{maintitle} means the title of the article where the book is edited. In our example \enquote{Un mémoire anonyme sur saint Barnabé (BHG 226e)}.
	\item \bibfield{pages} means the pages where the book is edited.
	\item \bibfield{substitle} means the subtitle of the edited book.
	\item \bibfield{title} means the title of the edited book. In our example \enquote{Mémoire sur le saint apôtre Barnabé}.


\end{itemize}


\subsection{Crossref's use}

You can also use the Biber's crossref's facilities. The \bibfield{crossfield} of a \bibtype{bookinarticle} entry refering to a \bibtype{article} entry. The fields are inherited from \bibtype{article} following these rule:

\begin{itemize}
	\item \bibfield{author} becomes \bibfield{bookauthor}.
	\item \bibfield{title} becomes \bibfield{maintitle}.
	\item \bibfield{subtitle} becomes \bibfield{maisubtitle}.

\end{itemize}

See the following example:

\inputminted{latex}{example.bib}



\subsection{\bibtype{inarticle} Entry Type}

The package also provides a \bibtype{inarticle} Entry Type, to show a section of an article with its own title. It's like \bibtype{bookinarticle}, but the \bibtype{title} field is printed with italic, and not with quotation marks.

\subsection{Loading package}

The package must be loaded after the \biblatex package:
\begin{minted}{latex}
\usepackage[…]{biblatex}
\usepackage{biblatex-bookinarticle}
\end{minted}

\subsection{Customization}

The way where \bibtype{bookinarticle} are printed is derivated from the \emph{verbose} bibliographic style. You can customize it by overriding bibliographic macros or bibliographic driver. Look at the file \verb+biblatex-bookinarticle+.

\section{Credits}

This package was created for Maïeul Rouquette's phd dissertation\footnote{\url{http://apocryphes.hypothese.org}.} in 2014. It is licensed on the \emph{\LaTeX\ Project Public License}\footnote{\url{http://latex-project.org/lppl/lppl-1-3c.html}.}. 


All issues can be submitted, in French or English, in the GitHub issues page\footnote{\url{https://github.com/maieul/biblatex-bookinarticle/issues}.}.


\section{Change history}


\begin{changelog}



\begin{release}{1.0.0}{2014-07-02}
\item First public release.
\end{release}
\end{changelog}
\end{document}

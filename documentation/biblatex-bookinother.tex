\documentclass{ltxdockit}[2011/03/25]
\usepackage{btxdockit}
\usepackage{fontspec}
\usepackage[mono=false]{libertine}
\usepackage{microtype}
\usepackage[american]{babel}
\usepackage[strict]{csquotes}
\setmonofont[Scale=MatchLowercase]{DejaVu Sans Mono}
\usepackage{shortvrb}
\usepackage{pifont}
\usepackage{minted}
\usepackage{graphics}
% Usefull commands
\newcommand{\biblatex}{\emph{biblatex}\xspace}
\pretocmd{\bibfield}{\sloppy}{}{}
\pretocmd{\bibtype}{\sloppy}{}{}
\newcommand{\namebibstyle}[1]{\texttt{#1}}
% Meta-datas
\titlepage{%
	title={Book edited in other other type with biblatex},
	subtitle={New data types},
	email={maieul <at> maieul <dot> net},
	author={Maïeul Rouquette},
	revision={2.0.0},
	date={09/03/2016},
	url={https://github.com/maieul/biblatex-bookinarticle}}

% biblatex
\usepackage[citestyle=verbose,bibstyle=bookinother]{biblatex}
\addbibresource{example-bookinarticle.bib}
\addbibresource{example-bookinincollection.bib}
\addbibresource{example-bookinthesis.bib}
\begin{document}

\printtitlepage
\tableofcontents

\section{Introduction}

\subsection{Aim}
The default \biblatex's styles provide a entry type called \bibtype{bookinbook}. 
However, it can happen, especially in classical philology, that a book is edited in other entry type. 
For example a book can be edited in article, in proceedings, in a thesis etc.
This packages provides new bibliographic entry type.

\subsection{History}

Originally, the package was called \emph{biblatex-bookinarticle}, because it provided only a new \bibtype{bookinarticle} entry type. 
However, many new types were added. 
Changing the name was required, and when the loading's way has changed, a good occasion happened.
\subsection{Credits}

This package was created for Maïeul Rouquette's phd dissertation\footnote{\url{http://apocryphes.hypothese.org}.} in 2014. It is licensed on the \emph{\LaTeX\ Project Public License}\footnote{\url{http://latex-project.org/lppl/lppl-1-3c.html}.}. 


All issues can be submitted, in French or English, in the GitHub issues page\footnote{\url{https://github.com/maieul/biblatex-bookinarticle/issues}.}.


\section{What does the package provide?}

The package provides:
\begin{itemize}
  \item New entry types. 
  \item Inheritance's mechanism for these entry types. 
  \item Integration of the entry types following the standard bibliography's styles of biblatex. 
\end{itemize}

It also provides new fields name for some entry type, and use the \emph{biblatex-morename} package to use new field.
 
\section{Loading package}
As the package defines new fields, you must load it as \verb+bibstyle+ option of \biblatex package.
 
\begin{minted}{latex}
  \usepackage[citestyle=verbose,bibstyle=bookinother]{biblatex}
\end{minted}

Notes that the \emph{bookintoher} bibliography's style automatically loads \emph{morenames} bibliography's style, which means its compatible with all the standard bibliography  styles of biblatex, because all of them are, in a way or an other, based on the \emph{verbose} bibliography's style, which is loaded by the \emph{morenames} bibliography's style.
 
In any case, you can choose you own citation's style.
 
If you need to use this package with package which also requires loading \emph{via} the \verb+bibstyle+ option, as for example biblatex-manuscript-philology, just use the biblatex-multiple-dm package, with the following way:

\begin{minted}{latex}
  \usepackage[tools={bookinother,manuscripts},bibstyle=verbose]{biblatex-multiple-dm}
  \usepackage[citestyle=verbose,bibstyle=multiple-dm]{biblatex}
\end{minted}
\section{The new entry types}

\subsection{\bibtype{bookinarticle} Entry Type}

A new entrytype is defined: \bibtype{bookinarticle}. It uses the standard fields of a \bibfield{article}, with these changes:

\begin{itemize}
	\item \bibfield{author} means the author of the ancient book.
	\item \textbf{\bibfield{bookauthor} means the author of the article where the book is edited, e.g. the modern editor of the book}.
	\item \bibfield{mainsubtitle} means the subtitle of the article where the book is edited.
	\item \bibfield{maintitle} means the title of the article where the book is edited. In our example \enquote{Un mémoire anonyme sur saint Barnabé (BHG 226e)}.
	\item \bibfield{pages} means the pages where the book is edited.
	\item \bibfield{subtitle} means the subtitle of the edited book.
	\item \bibfield{title} means the title of the edited book. In our example \enquote{Mémoire sur le saint apôtre Barnabé}.


\end{itemize}


\subsection{\bibtype{bookinincollection} Entry Type}

A new entrytype is defined: \bibtype{bookinincollection}. It uses the standard fields of a \bibfield{inincollection}, with these changes:

\begin{itemize}
	\item \bibfield{author} means the author of the ancient book.
	\item \textbf{\bibfield{bookauthor} means the author of the article where the book is edited, e.g. the modern editor of the book}.
	\item \bibfield{booksubtitle} means the subtitle of the article where the book is edited.
	\item \bibfield{booktitle} means the title of the article where the book is edited, in our example \enquote{\enquote{Non tibi proderit hec eruditio}. La versione latina degli \emph{Acta} greci del discepolo Tito}. 
	\item \bibfield{maintitle} means the title of the collection were the article is published, in our example \enquote{Suave mari magno\ldots}.
	\item \bibfield{mainsubtitle} means the subtitle of the collection were the article is published, in our example \enquote{studi offerti dai colleghi udinesi a Ernesto Berti}.
	\item \bibfield{pages} means the pages where the book is edited.
	\item \bibfield{subtitle} means the subtitle of the edited book.
	\item \bibfield{title} means the title of the edited book. In our example \enquote{Passio Sancti Titi Apostoli, Mense Ianurii die Quarto}.


\end{itemize}

\subsection{\bibtype{bookinthesis} Entry Type}

A new entrytype is defined: \bibtype{bookinthesis}. It uses the standard fields of a \bibfield{thesis}, with these changes:

\begin{itemize}
	\item \bibfield{author} means the author of the ancient book.
	\item \textbf{\bibfield{bookauthor} means the author of the thesis where the book is edited, e.g. the modern editor of the book}.
	\item \bibfield{booksubtitle} means the subtitle of the thesis where the book is edited.
	\item \bibfield{booktitle} means the title of the thesis where the book is edited. In our example \enquote{A Nice Title}.
	\item \bibfield{pages} means the pages where the book is edited.
	\item \bibfield{subtitle} means the subtitle of the edited book.
	\item \bibfield{title} means the title of the edited book. In our example \enquote{The Ancient Text}.


\end{itemize}

\subsection{Crossref's use}

You can also use the Biber's crossref's facilities.

\subsubsection{For \bibtype{bookinarticle}}
The \bibfield{crossfield} of a \bibtype{bookinarticle} entry refers to a \bibtype{article} entry. The fields are inherited from \bibtype{article} following these rules:

\begin{itemize}
	\item \bibfield{author} becomes \bibfield{bookauthor}.
	\item \bibfield{title} becomes \bibfield{maintitle}.
	\item \bibfield{subtitle} becomes \bibfield{mainsubtitle}.

\end{itemize}

See the following example:

\inputminted[breaklines]{latex}{example-bookinarticle.bib}

\subsubsection{For \bibtype{bookinincollection}}

The \bibfield{crossfield} of a \bibtype{bookinincollection} entry refers to a \bibtype{inincollection} entry. The fields are inherited from \bibtype{inincollection} following these rules:

\begin{itemize}
	\item \bibfield{author} becomes \bibfield{bookauthor}.
	\item \bibfield{booktitle} becomes \bibfield{maintitle}.
	\item \bibfield{booksubtitle} becomes \bibfield{mainsubtitle}.
	\item \bibfield{title} becomes \bibfield{booktitle}.
	\item \bibfield{subtitle} becomes \bibfield{booksubtitle}.

\end{itemize}

See the following example:

\inputminted[breaklines]{latex}{example-bookinincollection.bib}

\subsubsection{For \bibtype{bookinthesis}}
The \bibfield{crossfield} of a \bibtype{bookinthesis} entry refers to a \bibtype{article} entry. The fields are inherited from \bibtype{thesis} following these rules:

\begin{itemize}
	\item \bibfield{author} becomes \bibfield{bookauthor}.
	\item \bibfield{title} becomes \bibfield{booktitle}.
	\item \bibfield{subtitle} becomes \bibfield{booksubtitle}.

\end{itemize}

See the following example:

\inputminted[breaklines]{latex}{example-bookinthesis.bib}

\subsubsection{Summary}
The graphs~\ref{crossref} summaries the use of cross-referencing.

%\begin{figure}
%  \centering
%\includegraphics[height=0.99\textheight]{biblatex-bookinother-crossref.pdf}
%\label{crossref}
%\caption{Using crossref's mechanism with \emph{biblatex-bookinother}}
%\end{figure}

\subsection{\bibtype{inarticle} entry type}

The package also provides a \bibtype{inarticle} entry type, to show a section of an article with its own title. It is like \bibtype{bookinarticle}, but the \bibtype{title} field is printed with italic, and not with quotation marks.


\subsection{\bibtype{inincollection} Entry Type}

The package also provides a \bibtype{inincollection} entry type, to show a section of an article of a collection with its own title. It is like \bibtype{bookinincollection}, but the \bibtype{title} field is printed with italic, and not with quotation marks.

\subsection{\bibtype{inthesis} Entry Type}

The package also provides a \bibtype{inthesis} entry type, to show a section of a thesis with its own title. It is like \bibtype{bookinthesis}, but the \bibtype{title} field is printed with italic, and not with quotation marks.

The package also provides \bibtype{inphdthesis} and \bibtype{inmathesis}, similar to \bibtype{thesis}, with the \bibfield{type} field already defined.


\section{Customization}

The ways new entry types are printed are derivated from the \verb+standard.bbx+ bibliography's style. You can customize it by overriding bibliographic macros or bibliographic drivers. Look at the \verb+bookinother.bbx+file.

\section{Change history}


\begin{changelog}

\begin{release}{2.0.0}{2016-03-09}
\item Change name to \emph{biblatex-bookinother}.
\item Loading as a \biblatex bibliographic style and not more as a \LaTeX\ package.
\item Add \bibtype{bookincollection} and \bibtype{bookinproceedings}.
\item Add \bibtype{bookininproceedings} and related.
\end{release}

\begin{release}{1.3.0}{2016-02-11}
\item Add \bibtype{bookinthesis} and \bibtype{inthesis} and related.
\end{release}

\begin{release}{1.2.0a}{2016-02-07}
\item Fix handbook.
\end{release}

\begin{release}{1.2.0}{2016-02-05}
\item Formate   \bibfield{series}, \bibfield{volume} and \bibfield{number} fields of  \bibtype{inarticle} and \bibtype{bookinarticle} entries as \bibfield{series}, \bibfield{volume} and \bibfield{number} fields of \bibtype{article} entries.
\end{release}

\begin{release}{1.1.2}{2015-02-05}
\item Use the \bibfield{shortauthor} field to define the \bibfield{labelname} field (useful for some citation style, like authortitle).
\end{release}

\begin{release}{1.1.1}{2014-11-03}
\item Delete a false and not need test in the driver.
\item Compatibility with biblatex-dw family's styles.
\end{release}
\begin{release}{1.1.0}{2014-10-09}
\item Add \bibtype{bookinincollection} and \bibtype{inincollection}.
\end{release}

\begin{release}{1.0.0}{2014-07-02}
\item First public release.
\end{release}
\end{changelog}
\end{document}
